% ---
% RESUMOS
% ---

% resumo em português
\setlength{\absparsep}{18pt} % ajusta o espaçamento dos parágrafos do resumo
\begin{resumo}

This work proposes the usage of a VTOL (Vertical Take-Off and Landing) fixed wing aircraft for aerial photography and mapping. It entails the gathering of requisites, design, prototyping, and testing of the proposed UAV.


 \textbf{Palavras-chave}: otimização sem derivada, poços de petróleo, simulação, sintonia automática .
\end{resumo}

% resumo em inglês
\begin{resumo}[Abstract]
 \begin{otherlanguage*}{english}
Aerial mapping is one task that got revolutionized by the arrival of drones on the latest years. The manual job of taking pictures, printing and assembling them together was changed into putting coordinates into a software, and the pictures into another after the flight.

Depending of the task at hand, the operator can chose a multirotor for smaller areas, or a fixed-wing aircraft for larger ones. Both categories have their quirks: While multirotors are precise and can take-off/land virtually anywhere, their autonomy suffers as they generate all their lift by using propellers, Fixed-wing aircraft, on the other hand, can cover large areas quickly with a smaller power consumption, but are harder to position, and require larger areas for take-off an landing.
 
This work proposed an aircraft in between these two worlds. The prototype designed is a tail-sitting fixed-wing aircraft, able to take-off as a multirotor and transition into fixed-wing mode for more efficiency, enabling it to cover larger areas while needing a small area for take-off or landing and no additional apparatus for take-off.	

\todo{results!}

   \vspace{\onelineskip}
 
   \noindent 
   \textbf{Keywords}: derivative-free optimization, oil well, simulation, automatic tuning.
 \end{otherlanguage*}
\end{resumo}
