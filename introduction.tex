

\chapter{Introduction} \label{chap:Introduction}


\section{Novarum Sky}
	Novarum Sky is a still young company, created in 2014 and based in Florianópolis-Brazil. It develops aerial technologies for both manned and unmanned systems, including long-range digital audio/video transmission solutions, real time kinematics for precise localization during inspections and mapping, and inspections systems themselves.

The company was featured on Web Summit 2017 Lisboa, and has its main partners currently in Europe, with ongoing negotiations with MikroKopter and EDP.

\section{Motivation}
Technology and automation have been changing and improving many tasks on last few decades.
%
One of the tasks is aerial mapping, which started with balloons, then manned airplanes, and now, for smaller areas, is being increasingly done with drones\cite{dronesontherise}.
%

For the company, this project might mean a new innovative product, as it has both advantages of fixed-wing and rotary-wing aircraft.
Such product means there is no need for long landing stripes, nor for relatively expensive equipment such as landing parachutes. The competition is also low, as fixed wings are currently a niche market, compromising around 3\% of the photogrammetry solution by DroneDeploy\cite{dronedeployreport}.
%

In the context of Automation and Control Engineering, this project entails most of the areas discussed, such as mechanics, electronics, manufacturing, fast prototyping, and control of dynamic systems.
%	


\section{Objectives}

%
The final objective of the work is to have a working prototype of a VTOL fixed-wing UAV able to autonomously take off vertically, transition into fixed-wing mode, follow a planned path taking pictures, transition back into hover mode, and land autonomously.
%
It is planned to have a smaller prototype to test and tune the hover mode before testing the larger, heavier and more powerful final prototype, for safety and practicality reasons.
%
The possible on-board electronic flight controllers will be briefly described and one of them chosen.
%
An overview will be given of the control systems in place and their tuning.
%
The requisites for the job will be gathered, and the electromechanical structure designed and built around it.
%
It is expected that the prototype fulfills the hole between rotary-wing and fixed-wing aircraft by being able to land in tight spaces, while having a performance close to that of fixed-wing aircraft.

\section{What Is Out There?}
Right now the photogrammetry and mapping sector seems to be taken by multirotors, which compose 97\% of the drones registered on DroneDeploy\cite{dronedeployreport}, a cloud-based photogrammetry software. 

In Brazil, a few companies have fixed-wing aircraft for photogrammetry, such as Horus Aeronaves and Nuvem UAV. The drones however work only in fixed wing mode, and the larger ones (1.7 m on Horus' Verok, 2 m on Nuvem UAV's Batmap) need parachutes for landing. 

Internationally, Wingtra seems to be the state-of-the-art solution on tail-sitting VTOL aircraft. Its drone, WingtraOne, is able to take-off, fly and land autonomously, and comes bundled with all the required equipment and software required for the photogrammetry. The Wingtra project was born at the  Autonomous Systems Lab, at ETH Zurich in Switzerland, where a lot of research was done lately on such kinds of VTOL aircraft\cite{autonomoussystemslab}. The goal on this project is to get close to its performance and usability.
%%%%
%%
\section{Requisites}

For the design, a few conditions have been imposed by the available material and desired performance:

\begin{itemize}

\item The flight time should be atleast 1 hour.
\item The cruise speed should be around $15 m/s$.
\item The batteries used will be 6s lithium-polymer packs of 4500 mAh or 4s packs of 10000 mAh.
\item The motors should preferably be the ones already in use at the company, MK3538, MK3638, or MK3644
\item The UAV must be able to take of and land autonomously.

\end{itemize}


\section{Structure}
	
%
This report is structured in 9 chapters.
%
Chapter \ref{chap:Introduction} gives an introduction to the report.
%
Chapter \ref{chap:AerialMapping} describes the fields of aerial mapping and photogrammetry.
%
Chapter \ref{chap:FlightMechanics} delves into the flight mechanics and the UAV's mechanical design.
%
Chapter \ref{chap:electronics} shows the electronics involved.
%
Chapter \ref{chap:software} explains the involved software.
%
Chapter \ref{chap:control} shows the control structure and its tuning.
%
Chapter \ref{chap:prototyping} demonstrates the work to build the prototypes.
%
Chapter \ref{chap:assessment} details the validation process, the tests performed, and the results obtained.
%
Chapter \ref{chap:conclusions} closes with the conclusions.

%%%%%%%%%%%%%%%%%%%%%%%