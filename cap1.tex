

\chapter{Introduction} \label{chap:1}


\section{Novarum Sky}
	Novarum Sky is a still young company, created in 2014 and based in Florianópolis-Brazil. It develops aerial technologies for both manned and unmanned systems, including long-range digital audio/video transmission solutions, realtime kinematics for precise localization during inspections and mapping, and inspections systems themselves.

The company was featured on Web Summit 2017 Lisboa, and has it's main partners currently in Europe, with ongoing negotiations with MikroKopter and EDP.

\section{Motivation}
Technology and auomation have been changing and improving a lot of tasks on last few decades.
%
One of the tasks is aerial mapping, which started with balloons, then manned airplanes, and now, for smaller areas, is done mostly with drones\todo{citation needed, improvent needed}
%
For the company, this project might mean a new innovative product, as it has both advantages of fixed-wing and rotary-wing aircraft.
Such product means there's no need for long landing stripes, nor for relatively expensive equipment such as landing parachutes.
%
In the context of Automation and Control Engineering, this project entails most of the areas discussed, such as mechanics, electronics, manufacturing, fast prototyping, and control of dynamic systems.
%
%
%
% 
%
%


\section{Objectives}

%
The final objective of the work is to have a working prototype of a VTOL fixed-wing UAV able to autonomously take off vertically, transition into fixed-wing mode, follow a planned path taking pictures, transition into hover mode, and land autonomously.
%
It's planned to have a smaller prototype to test and tune the hover mode before testing the larger, heavier and more powerful final prototype, for safety and practicity reasons.
%
The possible on-board electronics will be briefly described and one of them chosen.
%
An overview will be given of the control systems in place and their tuning.
%
The requisites for the job will be gathered, and the eletro-mechanical structure designed and built around it.
%
It's expected that the prototype fulfills the hole between rotating-wing and fixed-wing aircraft by being able to land in tight spaces, but having a perfomance close to that of fixed-wing aircrafts. 

%%%%
%%
\section{Requisites}

For the design, a few conditions have been imposed by the available material and desired performance:

\begin{itemize}

\item The flight time should be between 1 and 2 hours.
\item The cruise speed must be around $15 m/s$.
\item The batteries used will be 6s lithium-polymer packs of 4500 mAh.
\item The motors should preferably be the ones already in use at the company, MK3538, Mk3638, or MK3644
\item The UAV must be able to take of and land autonomously.

\end{itemize}


\section{Structure}

%
This report is structured in 5 chapters.
%
Chapter 1 gives an introduction to the report.
%
Chapter 2 describes the fields of aerial mapping and photogrammetry.
%
Chapter 3 explains the requisites imposed on the aircraft.
%
Chapter 4 delves into the flight mechanics and the UAV's mechanical design.
%
Chapter 5 shows the electronics involved.
%
Chapter 6 shows the control structure and it's tuning.


%%%%%%%%%%%%%%%%%%%%%%%