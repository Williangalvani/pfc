

\chapter{Oil Well Modeling and Simulation} \label{chap:2}


\section{Offshore Oil Production}

The oilfields in question can have a range of different layouts. They will be presented and explained from bottom (of the sea) to top.

\subsubsection{Wellbore}
The "hole" itself connecting to the oil and gas reserves. It's design and architecture varies wildly with it's location (on-shore, off-shore) the composition of the reserves interfaces(rock properties, such as porosity).

\subsubsection{Wellhead}
The most upstream component, the wellhead sits on top of the wellbore, oil and gas reserves. It's the interface between the extraction systems and the reservoir.

\subsubsection{Choke}
It's used to manipulate the flow out of the wellhead.



\subsubsection{Manifold}

\subsubsection{Riser}

\subsubsection{Flow Lines}

\subsubsection{...}

\begin{itemize}

\item \textcolor{blue}{Discuss in the general terms the structure of an offshore oilfield, which consists of subsea oil wells with risers connecting to a production platform.}

\item \textcolor{blue}{Discuss in general terms the surface processing facilities, valves, separator, compressor, flare and exportation.}

\end{itemize}


\section{Satellite Oil Wells}

\textcolor{blue}{Discuss the structure of offshore oil wells that operate with continuous lift-gas injection (gas-lift).}



\section{Oil Well}


\begin{itemize}

\item \textcolor{blue}{Give a brief presentation about Marlim and/or Pipesim.}

\item \textcolor{blue}{Present some curves of oil wells modeled in Pipesim.}


\end{itemize}
