% ---
% RESUMOS
% ---

% resumo em português
\setlength{\absparsep}{18pt} % ajusta o espaçamento dos parágrafos do resumo
\begin{resumo}

Este trabalho propõe o uso de métodos de otimização sem derivadas na indústria de petróleo e gás, para sintonia de simuladores de poços de petróleo a partir de dados de poços reais. São analisados dois algoritmos, Nelder-Mead (Simplex) e OrthoMADS.
O simulador utilizado é o Pipesim, e para interface com ele são utilizados Python, Opal e pyWin32.
Os métodos são testados de forma a minimizar a distância quadrática entre a curva de produção de um poço real e a do poço simulado. São testadas também variações dos algoritmos.
O sistema proposto foi capaz de sintonizar os modelos desejados em tempo aceitável e com boa precisão.


 \textbf{Palavras-chave}: otimização sem derivada, poços de petróleo, simulação, sintonia automática .
\end{resumo}

% resumo em inglês
\begin{resumo}[Abstract]
 \begin{otherlanguage*}{english}
This work proposes the use of derivative-free optimization methods in the oil and gas industry, with the purpose of tuning an oil well simulators to match real oil wells. Two algorithms are analized, Nelder-Mead (simplex) and OrthoMADS.
The simulator used is Pipesim. Python, pyWin32 and Opal were used to interface with it.
The methods are tested by attempting to reduce the quadratic distance between the real well's production curve and the simulator's.
Some variations of the algorithms are also tested.
The proposed system was able to tune the simulator to a desired accuracy within an acceptable computing time.


   \vspace{\onelineskip}
 
   \noindent 
   \textbf{Keywords}: derivative-free optimization, oil well, simulation, automatic tuning.
 \end{otherlanguage*}
\end{resumo}
