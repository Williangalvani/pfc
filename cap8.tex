\chapter{Assessment} \label{chap:assessment}

The assessment was incremental. First, the aerodynamic properties were tested on a manual flight, qualitativaly, regarding properties such as stall angle, stall speed, and equilibrium point in flight. Following this, the hover capabilities were tested, such as altitude and atitude control. With the basic flight capabilities proven, a few autonomous, test flights were performed, without VTOL. Finally, it's VTOL capabilities were benchmarked. These tests are better described, as well as their results, in the following sections.


\section{Attitude Control Test}

To test the attitude control and stabilization, the prototype was hang by a rope, so it's range of movement was restricted, and it was safer to test indoors.
%
The first tests were qualitative. The wing was armed on the QStabilize mode, where the gyroscope and accelerometer are used to try to maintain the aircraft leveled in VTOL mode (propellers spinning parallel to ground).
%
The expected result was that the elevons should move trying to stop the movement, even without propellers on the motors (again, for safety reasons).
%
The aircraft successfully reacted to disturbances on it's attitude by moving the control surfaces appropriately.
%
The test was repeated with propellers, where it was noted that the roll response was sluggish. This problems troubleshooting is described in section \ref{sec:sluggish}.
