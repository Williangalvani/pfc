% ---
% RESUMOS
% ---

% resumo em português
\setlength{\absparsep}{18pt} % ajusta o espaçamento dos parágrafos do resumo
\begin{resumo}[Resumo]
 \begin{otherlanguage*}{portuguese}

Mapeamento aéreo é uma das tarefas que foi revolucionada com a chegada dos drones nos ultimos anos.
%
O trabalho manual de tirar fotos, organizá-las e juntá-las, mudou para colocar coordenadas em um software, e as fotos resultantes em outro para o pós-processamento após o vôo.
%

Dependendo da tarefa em questão, o operador pode escolher utilizar multirotores para áreas menores, ou uma aeronave de asa fixa para as maiores.
%
Enquanto multirotores são precisos e podem pousar/decolar de virtualmente qualquer lugar, sua autonomia sofre, uma vez que todo o empuxo para mantê-los em vôo é gerado diretamente pelas helices.
%
Aeronaves de asa fixa, por outro lado, podem cobrir grandes areas rapidamente com um consumo energético menor, mas são mais dificeis de posicionar e podem requerer dezenas de metros para pouso e decolagem.

%
Este trabalho propõe o desenvolvimento de uma aeronave entre estes dois mundos.
%
O protótipo projetado é uma aeronave de asa fixa \textit{tail-sitter}, capaz de decolar na vertical como um multirotor e transicionar para o modo de vôo asa fixa para maior eficiência, habilitando a cobertura de grandes áreas sem necessitar de aparatos adicionais para pouso e decolagem, nem de amplos espaços.
%

No teste realizado, foi comprovada a capacidade de decolagem e pousos verticais, no entanto não foi possível testar pousos e decolagens autônomos, tampouco transição e vôo horizontal, por limites de espaço e tempo. Apesar dos resultados parciais serem positivos, mais testes serão conduzidos até a finalização do produto.
%


 \textbf{Palavras-chave}: tail-sitter, aerofotogrametria, VANT.
  \end{otherlanguage*}
\end{resumo}

% resumo em inglês
\begin{resumo}[Abstract]

Aerial mapping is one task that got revolutionized by the arrival of drones on the latest years. The manual job of taking pictures, printing and assembling them together was changed into putting coordinates into a software, and the pictures into another after the flight.

Depending of the task at hand, the operator can chose a multirotor for smaller areas, or a fixed-wing aircraft for larger ones. Both categories have their quirks: While multirotors are precise and can take-off/land virtually anywhere, their autonomy suffers as they generate all their lift by using propellers, Fixed-wing aircraft, on the other hand, can cover large areas quickly with low	er power consumption, but are harder to position, and require larger areas for take-off and landing.
 
This work proposed an aircraft in between these two worlds. The prototype designed is a tail-sitting fixed-wing aircraft, able to take-off as a multirotor and transition into fixed-wing mode for more efficiency, enabling it to cover larger areas while needing a small area for take-off or landing and no additional apparatus for take-off.	

On the test performed, the VTOL capability was verified, however it was not possible to test autonomous take-offs and landings, nor transition and fixed-wing flight, due to time and space limitations. While the results observed are good when within the expected, more tests are required. 

   \vspace{\onelineskip}
 
   \noindent 
   \textbf{Keywords}: tail-sitter, aerophotogrammetry, UAV.

\end{resumo}
