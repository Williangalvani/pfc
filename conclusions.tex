\chapter{Conclusions} \label{chap:conclusions}

The proposed it as UAV system able to autonomously take-off, hover, transition into fixed-wing mode, fly autonomously over a pattern, transition back into hover mode, and land autonomously.

The proposed system was built in foam and tested.
The test performed involved only take-off and hovering. The aircraft behaved well on the harsh weather conditions, with 13 km/h winds.

While taking off and hovering are two of the trickiest parts, two crucial parts remain to be tested: Landing and transitioning, specially automatically.

The test had promising results, indicating that it's very much possible to have such an aircraft, but more tests are required to verify the remaining aspects.

The results, as they are, are deemed satisfactory.

For a mass-produced aircraft, however, a new landing-gear/winglet system should be thought of, one that is more tolerant to harsh lands, prevents the aircraft from tipping over, but still performs well aerodynamically.
The electronics bay should also be made of a stronger material than foam, mas even low-speed crashes can lose the components within it.
The motors should be better analyzed, and maybe changed for stronger ones, as while they seem to be under their maximum current (the wing currently uses 30 A on hover, meaning each motor is consuming around 15 A, leaving a 10 A margin for maneuverability), they are too hot to touch upon landing, which could be harmful to the motors.